% This is the start of every LaTeX document. The document class specifes the overall type of
% document you are writing. 'article' is most common. 'report' is similar to 'article', but allows
% for chapters. 'book' is for writin large books. 'beamer' for making presentations
\documentclass{article}

% Below are all the included packages. Packages add extra functionality to default LaTeX by adding
% new macros or changing default behaviours
\usepackage[margin=35mm]{geometry} % Controls layout of the page, including margins
\usepackage{biblatex}              % Controls bibliography and citations
\usepackage{listings}              % Source code layout, formatting and styling
\usepackage{color}                 % Needed for coloring source code syntax
\usepackage[hidelinks]{hyperref}   % Support for links. Autolinks citations, references, and TOC
\usepackage{amsmath}               % Maths equations and aligns
\usepackage{amssymb}               % More maths symbols

% Sets the title and author of the document. This is printed by \maketitle
% If date is not specified with \date{}, the current date is used
\title{\LaTeX{} Workshop}
\author{Tim Quelch}

% Add the references to the bibliography
\addbibresource{references.bib}

% Style options for souce code blocks (listing environment, and \lstinline{} macro)
\lstset{
	language=TeX,              % Language for syntax highlighting
	basicstyle=\ttfamily,      % Style for fonts
	numbers=left,              % Line numbers
	numberstyle=\tiny,         % Line number style
	frame=tb,                  % Frame lines on top and bottom
	tabsize=4,                 % Size of tabs
	columns=fixed,             % Fix character columns
	showstringspaces=false,    % Don't underscore spaces
	showtabs=false,            % Don't underscore tabs
	keepspaces,                % Keep spaces as spaces, not whitespace
	commentstyle=\color{red},  % Colour comments as red
	% keywordstyle=\color{blue},% Colour keywords as blue
	breaklines=true,           % Break lines, so they don't go over the page
}

% TODO config section numbers
% TODO config font selection
% TODO config fancy header

% All the actual content of the document goes in the document environment
\begin{document}

% Prints the title, author, and date.
\maketitle

% Prints the table of contents generated from the sections
\tableofcontents

% Inserts a page break (new page)
\newpage

\section{Introduction}
``\LaTeX{} is a high-quality typesetting system; it includes features designed for the production of technical and scientific documentation. \LaTeX{} is the de facto standard for the communication and publication of scientific documents. \LaTeX{} is available as free software.''
\cite{latex_project_latex_2018},

One of the key differences in \LaTeX{} to more common work processors (MS Word, LibreOffice etc.) is the separation of content and presentation. In \LaTeX{} the author describes the general structure of the document (section headings, paragraphs, equations, figures), and the layout and typesetting is handled by \LaTeX{} (or rather the underlying \TeX{} backend).

There are several advantages to this:
\begin{itemize}
  \item Author can focus on the actual content, rather than making everything look pretty
  \item Presentation can be changed easily without major formatting reworks throughout the document
  \item A standard(ish) format for submission to external publishers, who can then apply their own layout/presentation styles
\end{itemize}

\LaTeX{} is written in plain text, and is processed by an external program to generate the output files (usually a PDF)

\subsection{Pronounciation and Spelling}
\LaTeX{} is always \emph{lah-tech}, and sometimes \emph{lay-tech}. It is never pronounced \emph{lay-tecks}. It is always written with the \lstinline|\LaTeX{}| macro (output is the weird sized and spaced \LaTeX{}), or with the capitialistaion LaTeX.

\subsection{Basic Structure}
There are two major language structures that you will encounter when using \LaTeX{}: macros, and environments.

Macros are how we tell \LaTeX{} to do things. They take the general form of \lstinline{\commandname}, or with arguments \lstinline|\commandname[optional args]{required args}|. Macros can do anything from printing the table of contents (\lstinline{\tableofcontents}), \emph{emphasising} or \textbf{bolding} text (\lstinline|\emph{emphasising},\textbf{bolding}|), or even printing the funny \LaTeX{} logo (\lstinline|\LaTeX{}|). It is even possible to define your own macros to simplify things you are repeating often.

Environments are how we write larger sections of the document which often contain many lines or macros. Environments begin with a \lstinline|\begin{environmentname}| macro, and end with a \lstinline|\end{environmenname}| macro. Common environments include \lstinline{figure}, \lstinline{equation}, \lstinline{itemize} (these will be discussed later).

% TODO Preamble document environment

\section{Useful structure things} % TODO rename?

\subsection{Sections}
Sections are used to divide the document into parts. A new section is started with the \lstinline|\section{sectionname}| macro. Section titles are formatted to be bigger, bolded, and automatically numbered.

\subsection{Subsections}
We can also split sections into smaller \lstinline|\subsections{subsectionname}|
\subsubsection{Subsubsection}
... and \lstinline|\subsubsections{subsubsectionname}|

% TODO

% TODO snippe to turn off all numbering

\subsection{Lists}
% TODO

Dot point lists are created with the \lstinline{itemize} environment
\begin{itemize}
  \item Each item begins with the \lstinline{\item} macro
  \item another item
  \item another item
    \begin{itemize}
      \item You can also do nested lists by just starting a new \lstinline{itemize} environment
        \begin{itemize}
          \item And the marker changes for each level you are on
            \begin{itemize}
              \item List markers and spacing and spacing can be customised with the \lstinline{enumitem} package
            \end{itemize}
        \end{itemize}
    \end{itemize}
\end{itemize}


Numbered lists are created with the \lstinline{enumerate} environment
\begin{enumerate}
  \item Each item begins with the \lstinline{\item} macro
  \item another item
  \item another item
    \begin{enumerate}
      \item You can also do nested lists by just starting a new \lstinline{enumerate} environment
        \begin{enumerate}
          \item And the marker changes for each level you are on
            \begin{enumerate}
              \item List markers and spacing and spacing can be customised with the \lstinline{enumitem} package
            \end{enumerate}
        \end{enumerate}
    \end{enumerate}
\end{enumerate}

\section{Maths}

% TODO aligns, equations,
% TODO Text, greek symbols

\href{https://www.rpi.edu/dept/arc/training/latex/LaTeX_symbols.pdf}{The Great, Big List of \LaTeX{} Symbols} \cite{carlisle_great_2001}

\begin{equation}
    \alpha \beta \gamma \delta \epsilon \varepsilon \zeta \eta \theta \vartheta \iota \kappa \lambda \mu \nu \xi \pi \varpi \rho \varrho \sigma \varsigma \tau \upsilon \phi \varphi \chi \psi \omega
\end{equation}
\begin{equation}
    \Gamma \Delta \Theta \Lambda \Xi \Pi \Sigma \Upsilon \Phi \Psi \Omega
\end{equation}
% TODO Proof environment

\section{Figures, Tables, and Code}

% TODO

\subsection{References \& Labels}

% TODO

\subsection{\LaTeX{} knows better than you}

% TODO

\subsection{Build your own}

% TODO

\section{Citations}

% TODO

\printbibliography

\end{document}